% BEGIN TEMPLATE
\documentclass{article}
\usepackage{graphicx}
\usepackage{hyperref} 
\usepackage{xcolor}
\usepackage{nameref}
\usepackage{listings}
\usepackage{float}
\usepackage[title]{appendix}
\graphicspath{ {../../images/} }
\bibliographystyle{acm}
% CHANGE THESE
\newcommand{\courseListing}{CSCI 8110-001}
\newcommand{\courseName}{Advanced Machine Learning Applications}
\newcommand{\assignmentTitle}{Homework Assignment \#2}
\newcommand{\assignmentSubtitle}{Explainable Machine Learning}

\hypersetup{
    colorlinks,
    linkcolor={red!50!black},
    citecolor={blue!50!black},
    urlcolor={blue!80!black}
}
\urlstyle{same}
\definecolor{codegreen}{rgb}{0,0.6,0}
\definecolor{codegray}{rgb}{0.5,0.5,0.5}
\definecolor{codepurple}{rgb}{0.58,0,0.82}
\lstdefinestyle{mystyle}{
    commentstyle=\color{codegreen},
    keywordstyle=\color{magenta},
    numberstyle=\tiny\color{codegray},
    stringstyle=\color{codepurple},
    basicstyle=\ttfamily\footnotesize,
    breakatwhitespace=false,         
    breaklines=true,                 
    captionpos=b,                    
    keepspaces=true,                 
    numbers=left,                    
    numbersep=5pt,                  
    showspaces=false,                
    showstringspaces=false,
    showtabs=false,                  
    tabsize=2
}

\lstset{style=mystyle}

\begin{document}
  \input{../../templates/titlepage.tex}
  \graphicspath{{./images/}}
% END TEMPLATE

\section{Project Setup}
\par Similar to the prior assignment, all work in this project was developed and executed using the Paperspace Gradient service.
Gradient allows for rapid setup of TensorFlow notebooks on GPU-powered machines.


\begin{appendices}

% \newpage
% \section{Complete Code Listing} \label{codelist}
% \lstinputlisting[language=Python]{HW1_code.py}
\end{appendices}

\begin{thebibliography}{9}
  \bibitem{KerasLayers} 
  Adrian Rosebrock. 2018. Keras Conv2D and Convolutional Layers. (December 2018). Retrieved September 21, 2020 from https://www.pyimagesearch.com/2018/12/31/keras-conv2d-and-convolutional-layers/.
  
  \end{thebibliography}

\end{document}