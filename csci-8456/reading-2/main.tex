% BEGIN TEMPLATE
\documentclass{article}
\usepackage{graphicx}
\usepackage{hyperref} 
\usepackage{xcolor}
\graphicspath{ {../../images/} }
\bibliographystyle{acm}
% CHANGE THESE
\newcommand{\courseListing}{CSCI 8456-001}
\newcommand{\courseName}{Introduction to Artificial Intelligence}
\newcommand{\assignmentTitle}{The Winograd Schema Challenge}
\newcommand{\assignmentSubtitle}{Reading on AI \#2}

\hypersetup{
    colorlinks,
    linkcolor={red!50!black},
    citecolor={blue!50!black},
    urlcolor={blue!80!black}
}
\urlstyle{same}

\begin{document}
  \input{../../templates/titlepage.tex}
  \graphicspath{{./images/}}
% END TEMPLATE
\par One interesting idea gleaned from this paper is the idea of setting acceptable testing and evaluation parameters, around which a better definition of intelligence can be formed.
The authors of this paper posit that the Turing Test, generally regarded as the aspirational standard for assessing machine intelligence, is not complete.
Their rationale is that the idea of "conversational" machines is not easily quantifiable--humans are either too lenient in defining what is or is not a "conversation," or machines are able to "cheat" by using canned, context-independent responses in a way that evades meaningful assessment.
Their test, the Winograd Schema challenge, is a solution to this in that it provides short, context aware challenges that have to be logically solved by a system being assessed.

\par Like the previous paper, authored by John McCarthy, this paper continues to challenge my assumptions about the idea of "intelligent" machines not by \textit{building} an intelligent machine but by asking me to evaluate how, if at all, I might consider a machine to be intelligent.
If we aren't able to identify the reasons why systems display intelligence, or if we can't establish repeatable, measurable results in AI testing, we aren't really performing computer \textit{science}.
From an academic perspective, keeping this in mind will be extremely useful because it forms the basis upon which new ideas can be built.
Even in industry, though, well-defined problems and applications are useful because they prevent software engineers from wasting resources in order to chase nebulous results.

\end{document}Even in industry, though, well-defined problems and applications are useful because they prevent software engineers from wasting resources in order to chase nebulous results.

