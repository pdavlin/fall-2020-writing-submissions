% BEGIN TEMPLATE
\documentclass{article}
\usepackage{graphicx}
\usepackage{hyperref} 
\usepackage{xcolor}
\graphicspath{ {../../images/} }
\bibliographystyle{acm}
% CHANGE THESE
\newcommand{\courseListing}{CSCI 8456-001}
\newcommand{\courseName}{Introduction to Artificial Intelligence}
\newcommand{\assignmentTitle}{Reading on AI: Classics}
\newcommand{\assignmentSubtitle}{Reading on AI \#1}

\hypersetup{
    colorlinks,
    linkcolor={red!50!black},
    citecolor={blue!50!black},
    urlcolor={blue!80!black}
}
\urlstyle{same}

\begin{document}
  \input{../../templates/titlepage.tex}
  \graphicspath{{./images/}}
% END TEMPLATE
\par The idea that was most engaging in this article and, indeed, the one that McCarthy seems to continually return to throughout the paper, is the idea that the way \textit{humans} define intelligence is limited in ways that we cannot yet quantify. 
McCarthy says, in particular, "we understand some of the mechanisms of intelligence and not others." 
In particular, McCarthy's explanation that the Turing Test is one-sided, and that a machine which fails that test lends better academic perspective to many of the news articles that I've seen about AI. 
\par These articles where, for example, an AI program defeats a world-champion \textit{Go} player using an unprecedented sequence of moves \cite{GoAiWired} are instances, in my mind, where these "mechanisms of intelligence" that McCarthy mentions begin to break down. 
Both speak to why AI, as a field, is so exciting to me as a student--engaging in the pursuit of intelligence that we can't yet quantify.
\\[0.5in]

\begin{thebibliography}{9}
  \bibitem{GoAiWired} 
  Cade Metz. 2017.
  \textit{The Sadness and Beauty of Watching Google's AI Play Go}. 
  Retrieved August 27, 2020 from \url{https://www.wired.com/2016/03/sadness-beauty-watching-googles-ai-play-go/}.
  
  \end{thebibliography}

\end{document}